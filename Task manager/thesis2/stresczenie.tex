\documentclass[12pt]{report}
\usepackage[T1]{fontenc}
\usepackage[utf8]{inputenc}
\usepackage{caption}
\usepackage{adjustbox}
\usepackage{graphicx}
\usepackage{amsmath,amssymb,amsfonts}
\usepackage{txfonts}
\usepackage{listings}
\usepackage{float}
\usepackage{color}
\usepackage{xcolor}
\usepackage{enumitem}
\graphicspath{{images/}}
\renewcommand{\chaptername}{Rozdział}
\renewcommand{\contentsname}{Spis treści}
\renewcommand{\figurename}{Rysunek}
\renewcommand{\listfigurename}{Spis rysunków}
\renewcommand{\bibname}{Bibliografia}
\setlength{\textwidth}{14cm}
\setlength{\textheight}{20cm}
\newtheorem{definition}{Definicja}
\newtheorem{example}{Przykład}[chapter]
\lstdefinelanguage{JavaScript}{
  keywords={typeof, new, true, false, catch, function, return, null, catch, switch, var, if, in, while, do, else, case, break},
  keywordstyle=\color{blue}\bfseries,
  ndkeywords={class, export, boolean, throw, implements, import, this},
  ndkeywordstyle=\color{darkgray}\bfseries,
  identifierstyle=\color{black},
  sensitive=false,
  comment=[l]{//},
  morecomment=[s]{/*}{*/},
  commentstyle=\color{purple}\ttfamily,
  stringstyle=\color{red}\ttfamily,
  morestring=[b]',
  morestring=[b]"
}
\lstset{
   language=JavaScript,
   backgroundcolor=\color{lightgray},
   extendedchars=true,
   basicstyle=\footnotesize\ttfamily,
   showstringspaces=false,
   showspaces=false,
   numbers=left,
   numberstyle=\footnotesize,
   numbersep=9pt,
   tabsize=2,
   breaklines=true,
   showtabs=false,
   captionpos=b
}
\lstdefinelanguage{XML}
{
  morestring=[b]",
  morestring=[s]{>}{<},
  morecomment=[s]{<?}{?>},
  stringstyle=\color{black},
  identifierstyle=\color{blue},
  keywordstyle=\color{cyan},
  morekeywords={xmlns,version,type}% list your attributes here
}
\lstset{
  basicstyle=\ttfamily,
  columns=fullflexible,
  showstringspaces=false,
  commentstyle=\color{gray}\upshape
}
\colorlet{punct}{red!60!black}
\definecolor{background}{HTML}{EEEEEE}
\definecolor{lightgray}{rgb}{.9,.9,.9}
\definecolor{darkgray}{rgb}{.4,.4,.4}
\definecolor{purple}{rgb}{0.65, 0.12, 0.82}
\definecolor{delim}{RGB}{20,105,176}
\colorlet{numb}{magenta!60!black}
\lstdefinelanguage{json}{
    basicstyle=\normalfont\ttfamily,
    numbers=left,
    numberstyle=\scriptsize,
    stepnumber=1,
    numbersep=8pt,
    showstringspaces=false,
    breaklines=true,
    frame=lines,
    backgroundcolor=\color{background},
    literate=
     *{0}{{{\color{numb}0}}}{1}
      {1}{{{\color{numb}1}}}{1}
      {2}{{{\color{numb}2}}}{1}
      {3}{{{\color{numb}3}}}{1}
      {4}{{{\color{numb}4}}}{1}
      {5}{{{\color{numb}5}}}{1}
      {6}{{{\color{numb}6}}}{1}
      {7}{{{\color{numb}7}}}{1}
      {8}{{{\color{numb}8}}}{1}
      {9}{{{\color{numb}9}}}{1}
      {:}{{{\color{punct}{:}}}}{1}
      {,}{{{\color{punct}{,}}}}{1}
      {\{}{{{\color{delim}{\{}}}}{1}
      {\}}{{{\color{delim}{\}}}}}{1}
      {[}{{{\color{delim}{[}}}}{1}
      {]}{{{\color{delim}{]}}}}{1},
}

\begin{document}
\section*{Streszczenie}
Szerzący się globalny dostęp do internetu zaowocował potrzebą jak najbardziej efektywnego działania serwerów, zdolnych do obsługi wielu klientów równolegle w jak najkrótszym czasie. 
Odpowiedzią na te potrzeby jest powstanie technologii Node.js, która dzięki prostej obsłudze, wysokiej wydajności oraz budowie wykorzystującej technologie wielowątkowe umożliwia tworzenie zarówno nieskomplikowanych, jak i złożonych aplikacji webowych. 
Praca opisuje przeznaczenie, budowę, schemat działania serwera oraz technikę wykonania i prezentację przykładowej aplikacji służącej do zarządzania codziennymi zadaniami. 
Prezentuje pełny projekt zawierający wymagania, sposoby rozwiązania poszczególnych problemów oraz moduły technologii Node.js wykorzystane w procesie tworzenia aplikacji. 
Zostały przeprowadzone testy manualne, sprawnościowe, obciążeniowe oraz wydajnościowe wykonanego programu, a wyniki podsumowane i zaprezentowane. 
\newline
Praca stanowi swoistą bazę wiedzy na temat technologii Node.js oraz przedstawia pełny proces twórczy przykładowej aplikacji wykorzystującej jej możliwości. 

\section*{Abstract}
Spreading global access to the internet creates the need of the most effective-working servers that could handle huge amount of simultaneous clients in as short as possible time.
The answer to this need is Node.js technology which is easy to use, high efficient and it uses multithread technology that provides possibilities to create both simple and complex web applications. 
The thesis describes purpose, architecture, server working scheme and implementation techniques as well as presentation of working application to manage daily tasks.
It presents full description of web application project containing requirements, problem-solving methods and different Node.js technology modules. 
There have been done manual, efficiency, load and performance tests and their results are summarized and presented. 
\newline
This thesis contains base knowledge about Node.js technology and presents full productive process of sample projects, that use it possibilities.


\end{document}