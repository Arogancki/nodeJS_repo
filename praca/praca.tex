\documentclass[12pt]{report}
\usepackage[T1]{fontenc}
\usepackage[utf8]{inputenc}
\usepackage{graphicx}
\usepackage{amsmath,amssymb,amsfonts}
\usepackage{txfonts}
\graphicspath{ {images/} }
\renewcommand{\chaptername}{Rozdział}
\renewcommand{\contentsname}{Spis treści}
\renewcommand{\figurename}{Rys.}
\renewcommand{\tablename}{Tab.}
\renewcommand{\listfigurename}{Spis rysunków}
\renewcommand{\listtablename}{Spis tabel}
\renewcommand{\bibname}{Bibliografia}
\pagestyle{headings}
\setlength{\textwidth}{14cm}
\setlength{\textheight}{20cm}
\newtheorem{definition}{Definicja}
\newtheorem{example}{Przykład}[chapter] 
\newtheorem{corollary}{Wniosek}[chapter]

\begin{document}
\begin{titlepage}
	\centering
	{\scshape\LARGE Celem pracy jest przedstawienie technologii Node.js na przykładzie zaprojektowaniej aplikacji przeznaczonej do zarządzania codziennymi zadaniami.  \par}
\end{titlepage}

\tableofcontents


\chapter{Wstęp} \label{rozdz.wstep} 
{\em Praca MUSI stanowić samodzielne opracowanie przez dyplomanta WYBRANEGO TE\-MATU



\addcontentsline{toc}{chapter}{Bibliografia} 
\begin{thebibliography}{99}
\bibitem{Brown} 
Ethan Brown
\textit{Web Development with Node and Express: Leveraging the JavaScript Stack 1st Edition ISBN: 9781491949306}
\bibitem{Onodi} 
Robert Onodi
\textit{MEAN Blueprints ISBN: 9781783553945}
\bibitem{Nodejs} 
Dokumentacja języka programowania node.js
\textit{https://nodejs.org/en/docs}

\end{thebibliography}

\addcontentsline{toc}{chapter}{Spis rysunków} 
\listoffigures
\addcontentsline{toc}{chapter}{Spis tabel}

\end{document}
