\documentclass[12pt]{report}
\usepackage[T1]{fontenc}
\usepackage[utf8]{inputenc}
\usepackage{caption}
\usepackage{adjustbox}
\usepackage{graphicx}
\usepackage{amsmath,amssymb,amsfonts}
\usepackage{txfonts}
\usepackage{listings}
\usepackage{float}
\usepackage{color}
\usepackage{xcolor}
\usepackage{enumitem}
\graphicspath{{images/}}
\renewcommand{\chaptername}{Rozdział}
\renewcommand{\contentsname}{Spis treści}
\renewcommand{\figurename}{Rysunek}
\renewcommand{\listfigurename}{Spis rysunków}
\renewcommand{\bibname}{Bibliografia}
\setlength{\textwidth}{14cm}
\setlength{\textheight}{20cm}
\newtheorem{definition}{Definicja}
\newtheorem{example}{Przykład}[chapter]
\lstdefinelanguage{JavaScript}{
  keywords={typeof, new, true, false, catch, function, return, null, catch, switch, var, if, in, while, do, else, case, break},
  keywordstyle=\color{blue}\bfseries,
  ndkeywords={class, export, boolean, throw, implements, import, this},
  ndkeywordstyle=\color{darkgray}\bfseries,
  identifierstyle=\color{black},
  sensitive=false,
  comment=[l]{//},
  morecomment=[s]{/*}{*/},
  commentstyle=\color{purple}\ttfamily,
  stringstyle=\color{red}\ttfamily,
  morestring=[b]',
  morestring=[b]"
}
\lstset{
   language=JavaScript,
   backgroundcolor=\color{lightgray},
   extendedchars=true,
   basicstyle=\footnotesize\ttfamily,
   showstringspaces=false,
   showspaces=false,
   numbers=left,
   numberstyle=\footnotesize,
   numbersep=9pt,
   tabsize=2,
   breaklines=true,
   showtabs=false,
   captionpos=b
}
\lstdefinelanguage{XML}
{
  morestring=[b]",
  morestring=[s]{>}{<},
  morecomment=[s]{<?}{?>},
  stringstyle=\color{black},
  identifierstyle=\color{blue},
  keywordstyle=\color{cyan},
  morekeywords={xmlns,version,type}% list your attributes here
}
\lstset{
  basicstyle=\ttfamily,
  columns=fullflexible,
  showstringspaces=false,
  commentstyle=\color{gray}\upshape
}
\colorlet{punct}{red!60!black}
\definecolor{background}{HTML}{EEEEEE}
\definecolor{lightgray}{rgb}{.9,.9,.9}
\definecolor{darkgray}{rgb}{.4,.4,.4}
\definecolor{purple}{rgb}{0.65, 0.12, 0.82}
\definecolor{delim}{RGB}{20,105,176}
\colorlet{numb}{magenta!60!black}
\lstdefinelanguage{json}{
    basicstyle=\normalfont\ttfamily,
    numbers=left,
    numberstyle=\scriptsize,
    stepnumber=1,
    numbersep=8pt,
    showstringspaces=false,
    breaklines=true,
    frame=lines,
    backgroundcolor=\color{background},
    literate=
     *{0}{{{\color{numb}0}}}{1}
      {1}{{{\color{numb}1}}}{1}
      {2}{{{\color{numb}2}}}{1}
      {3}{{{\color{numb}3}}}{1}
      {4}{{{\color{numb}4}}}{1}
      {5}{{{\color{numb}5}}}{1}
      {6}{{{\color{numb}6}}}{1}
      {7}{{{\color{numb}7}}}{1}
      {8}{{{\color{numb}8}}}{1}
      {9}{{{\color{numb}9}}}{1}
      {:}{{{\color{punct}{:}}}}{1}
      {,}{{{\color{punct}{,}}}}{1}
      {\{}{{{\color{delim}{\{}}}}{1}
      {\}}{{{\color{delim}{\}}}}}{1}
      {[}{{{\color{delim}{[}}}}{1}
      {]}{{{\color{delim}{]}}}}{1},
}

\begin{document}

\tableofcontents

\chapter{Wstęp}
\chapter{Kryteria wyboru freameworkow}
\chapter{Kryteria oceny}
- oceny od 1 do 3 1 - wogole nie wspiera trzeba uzyc osobnego narzedzia / wolne / wymaga duzo pracy czym wiecej kodu tym ciezej / cieszkie opisy przyswajanie
2 - wspiera czesciowo jednak trzeba skonfiguraowac narzedzie / srednie / wymaga noramlnie kodu. nie rosnie z czasem / ok oopisy zrozuumiale
3 - w pelni wspiera lub minimalny efort w celu dzialania / bardzo predko / czym wiecej juz mam napisane tym mniej mam pisac / super dokumetacja wsparcie testowe kody i zloto

- integracja z warstwa widoku
- integracja z baza danych
- szybkosc przy rownoleglych zapytaniach
- szybkosc przy kolejnych zapytaniach
- rozszerzalnosc aplikacji
- testowanie
- popularnosc (w tym czy jest utrzymywana)
- dokumentacja
- prog wejscia
- ilosc kodu 
- ocena kryteriow - ktore na ile wazne, okreslenie wag kryteriow

\chapter{Opis wybranych rozwiązań}
Sails
ExpressJS
Meteor
\chapter{Analiza porównawcza}
- dla kazdej technologi o kazdym kryterium oceny
\chapter{Podsumowanie}
- zestawienie punktacji dla kazdego
- ktore do czego najlepiej a do czego najgorzej sie nadaje
- osobisty faworyt

\addcontentsline{toc}{chapter}{Bibliografia} 
\begin{thebibliography}{99}
\bibitem{Brown}
E. Brown
\textit{"Web Development with Node and Express", 2014}
\bibitem{Mardan}
Azat Mardan
\textit{"Express.js Guide: The Comprehensive Book on Express.js", 2016}
\bibitem{StrongLoop}
StrongLoop
\textit{dokumentacja Express, 2017, źródło: https://expressjs.com/en/4x/api.html}
\bibitem{McNeil&Nathan}
Mike McNeil, Irl Natha
\textit{"Sails.js in Action", 2017}
\bibitem{Shahid}
Shahid Shaikh
\textit{"Sails.js Essentials", 2016}
\bibitem{McNeil}
Mike McNeil
\textit{dokumentacja Sails, 2017, źródło: https://sailsjs.com/documentation/reference}
\bibitem{Strack}
Isaac Strack
\textit{"Getting Started with Meteor JavaScript Framework", 2012}
\bibitem{Vogelsteller&Strack&Reyna}
Fabian Vogelsteller, Isaac Strack, Marcelo Reyna
\textit{"Meteor: Full-Stack Web Application Development", 2016}
\bibitem{MDG}
Meteor Development Group
\textit{dokumentacja Meteor API, 2017, źródło: https://docs.meteor.com/#/full/}
\bibitem{Wolff}
Eberhard Wolff
\textit{"Microservices: Flexible Software Architecture", 2016}
\bibitem{Zeidman}
Bob Zeidman
\textit{"The Software IP Detective's Handbook: Measurement, Comparison, and Infringement Detection", rok}
\bibitem{Onodi}
Node.js Foundation
\textit{dokumentacja języka programowania Node.js, 2017, źródło: https://nodejs.org/en/docs/}
\end{thebibliography}
\addcontentsline{toc}{chapter}{Spis rysunków} 
\listoffigures

\end{document}



























